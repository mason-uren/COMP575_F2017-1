\textit{Multi Agent Localization from Noisy Relative Pose Measurements}~\cite{aragues2011multi} \\

Within the paper, the writers address the problem of estimating rovers' poses who do not share any common reference frames, when working with multi-agent systems. Their goal, in the presence of added noise, is to compute each agents pose, either relative to a specified anchor node or exclusively using neighbor-to-neighbor or local interations. The setup is to arrange independent rover interaction (poses) into a directed graph and have an anchor node placed at pose (0,0,0). Localization then is achieved in three phases: 1) an estimate of all the node orientations $\hat{\theta} \in \mathbb{R}^{n-1}$ in realation to the anchor, 2) then tangential position measurements of the nodes are expressed in terms of previuos iterations, 3) finally estimate pose positions and orientations are calculated using \textbf{p*} $= ( (\textbf{x*})^T, (\theta^*)^T )^T$. Results proved favorable as both implementations converged to values within one standard deviation of the actual poses. An interesting aside found that the distributive approach converged under a wider range of classes of communications graphs.

Localization is extremely helpful when navigating in an unknown environment. This approach suggests accuracy of within centimeters, which means, in relation to my project, the rovers can have greater confidence in goal, resource, and holding positions. We can also wrap up rover-to-rover avoidance by relaying relative positions to one another.