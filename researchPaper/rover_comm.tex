\textit{Aaladdin: a Meta-Model for Analysis and Design of Organizations in Multi-Agent Systems}~\cite{ferber1998meta} \\

The authors seak to create a interlinked rover communication network based around a two tiered obstraction model. They propose a $concrete~level$, composed of the physical organization of the rover's core concepts (agents, groups, and role), and a second tier or $abstract/methodological~level$, which defines all possible roles, valid interactions, and structure of groups and organizations. At the $concrete~level$, agents are a member of a group, group contains a role, and role is handled by agents. While on the $abstract~level$, group is contained by an organization which is instantiated from an organization structure, organization structure holds a group structure which can then instantiate to a group or define an interaction, interaction is defined between roles, and an agent is instantiated from an agent class. Tests were then run on an implemented "Mulit-Agent Development Kit" (MadKit) where they successfully validate their approach. By sticking to a basic philosophy, that the platform should handle its own management when working with agents, organized into groups, identified by roles, that the system supported distribution, multi-agents, and is scalable.

This concrete, yet complex communication and permission system architecture scales well to my research. Interlinked and "cognitive" rovers are the back-bone of multi-agent systems, thus by having a two tiered model built on simplicity and abstraction, we can confidently create any style of organization based around agents, groups, and roles. This gives us the freedom to tailor the model architecture to our needs.