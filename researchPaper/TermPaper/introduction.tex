Compexity of multi-agent systems can be broken down into two pieces: 1) clear and current inter-agent communication, and 2) percision agent localization about a global point of reference. To successfully accomplish a group task, each agent needs to have relavant, accurate infromation about it's neighbors and be able to correctly determine its own location. Characterized by low-cost, high-quantity demands, the challenge surrounding swarm robotics has thus been how to maintain high performance with economical sensors that are prone to error. These observations prompted us to develop, then evaluate alorgithms that seek to eliminate noise within data feedback. To this end, we propose a unique approach for dynamically calculating agent location within a multi-agent network.

Concise communication across a field of agents, boils down to the structure of the proposed model and the alrgorithm that decides when and what to communicate. 

The first is discussed by Ferber et al. in \cite{ferber1998meta}, where the authors seak to create a interlinked rover communication network based around a two tiered obstraction model. They propose a concrete level, composed of the physical organization of the rover’s core concepts (agents, groups, and role), and a second tier or abstract/methodological level, which defines all possible roles, valid interactions, and structure of groups and organizations. At the concrete level, agents are a member of a group, group contains a role, and role is handled by agents. While on the abstract level, group is contained by an organization which is instantiated from an organization structure, organization structure holds a group structure which can then instantiate to a group or define an interaction, interaction is defined between roles, and an agent is instantiated from an agent class. Thus by having a two tiered model built on simplicity and abstraction, we can confidently create any style of organization based around agents, groups, and roles.

The later, examines the threshold that decides when it is most appropriate to $send$ or $not-send$ information. The authors \cite{becker2009analyzing} establish an inter-rover communication thresh- old, which decides when to communicate information, when constrained by partial rover maps of their overall world. Communication actions are simply $communicate$ or $don′t~communicate$, where if one agent chooses to communicate, all agents broadcast their local state to eachother at some cost. This synchronizes the world view generated by the rovers, providing complete information about the current world state. They develop a communication protocol, $Model~Lookahead$, that only requires the rovers to hold the latest message of their projected world, instead of saving the entire message history, which ultimately saves times computationaly. Introducing and controlling messages in this way proved favorable as overcommunication of redundant information was decreased. In two different experimental setups, this approach produced smooth and monotonous degradation of values as communication costs increased.

Beyond comminication, navigating within any environment requires the estimation of an agents current position in reference to a global anchor. Analysed by Aragues et al. \cite{aragues2011multi}, the writers address the problem of estimating rovers’ poses who do not share any common reference frames, when working with multi-agent systems. Their goal, in the presence of added noise, is to compute each agents pose, either relative to a specified anchor node or exclusively using neighbor- to-neighbor or local interations. The setup is to arrange independent rover interaction (poses) into a directed graph and have an anchor node placed at pose (0,0,0). Localization then is achieved in three phases: 1) an estimate of all the node orientations $\hat{\theta} \in \mathbb{R}^{n-1}$ in realation to the anchor, 2) then tangential position measurements of the nodes are expressed in terms of previuos iterations, 3) finally estimate pose positions, \textbf{p}, and orientations, \textbf{x}, are calculated using $\textbf{p*} = ((\textbf{x*})^T , (\theta∗)^T )^T$ . Results proved favorable as both implementations converged to values within one standard deviation of the actual poses. An interesting aside found that the distributive approach converged under a wider range of classes of communications graphs.

In summary, the main contribution to this paper is to provide algorithms that correct the inherint drift within GPS units and establish a stable communication network between multiple agents. To achieve this, localization algorithms must collectively create and agree on a global anchor node for the agents to reference, while communicating only pertinant environmental information to the group members, reducing redundant messages.

The remainder of this paper is organized as follows. Section~\ref{prob} provides a mathematical description of the sensor models and the localization problem, which is followed in Section~\ref{exp} by an description of the methods used. Next analysis and experimental results are detailed in Section~\ref{results}, followed in Section~\ref{concl} by the final conclusions and directions for future work. 