To test the validity of our proposed algorithms we needed a way to mimic the error inherit within cost-effective GPS units. Using python, we modeled the drift form our actual position to our perceived location $X_{GPS}$, seen below:

\begin{equation}
	X_{GPS} = 
	\begin{cases}
	(x_c, R_{max}, d_{max}) & \text{On first iteration} \\
	(x_p, R_{max}, d_{max}) & \text{otherwise}
	\end{cases}
	\label{eq:genorator}
\end{equation}

where $x_c$ is the origin pose, $x_p$ is the previously generated error pose, $R_{max}$ creates the min/max bounds that dictate how far the rover can drift, and $d_{max}$ is the range the rover is allowed to drift in any direction at each iteration. 

Using eq.~\ref{eq:genorator}, we can create relative agent position for both the static and dynamic cases. When dealing with a stationary agents, we simply set $x_p$ to $x_c$ and then proceed to the next iteration, but if the rover is moving we cannot do a straight substitution. Instead we need to factor in the goal position of the rover $x_g$ in combination with the timestep between sensor readings $t$ to calculate a change in distance $\triangle x$, thus $\triangle x = x_g \cdot t$. We then add the difference to $x_c$ and then substitute the final value into $x_p$, as seen below:

\begin{equation}
x_p = 
	\begin{cases}
		x_c & \text{static} \\
		x_c + \triangle x & \text{dynamic}\\
	\end{cases}
	\label{eq:genorator_expln}
\end{equation}

Key to our algorithms is the ability to correct the induced error, adjusting the percieved pose by an offset. Assuming that $w$ is characterized by:

\[
w = 
\begin{cases}
	1 & \text{if} d_a < 0.1 \\
	e^{-d_a} & \text{otherwise} \\
\end{cases}
\]

where $d_a$ is the agent's distance to the anchor node,  $u$ is model with:

\[
\begin{cases}
	1 & \text{if} d_g < 0.05 \\
	\frac{1}{(1 + d_g)} & \text{otherwise} \\
\end{cases}
\]

such that $d_g$ is the agent's distance to the goal pose, and $v$ follows:

\[
\begin{cases}
	0 & \text{if} \frac{\delta x}{\delta t} < 0.01 \\
	\frac{\delta x}{\delta t} \cdot 0.1 & \text{otherwise}
\end{cases}
\]

then we can say, 

\begin{equation}
	x_c = w \cdot (x_c - x_p) + u \cdot (x_g - x_c) + v \cdot cos/sin(\theta)
	\label{eq:dy_local}
\end{equation}

such that $x_c, x_p, x_g$ are current, previous, and goal positions respectively, and $cos/sin$ is used when considering $x$ or $y$ respectively. This allows us to account for both the static and dynamic cases of agent localization where, $w \cdot (x_c - x_p)$ corrects GPS drift, $u \cdot (x_g - x_c)$ corrects odometry error, and $v \cdot cos/sin(\theta)$ looks to see if the rover is moving and in which direction.
