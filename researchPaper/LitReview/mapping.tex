\textit{Distributed Localization and Mapping with a Robotic Swarm}~\cite{rothermich2004distributed} \\

The writers discuss the issues native to location and virtual mapping within swarm robotics, when working with limited sensing and communication. Their work is motivated by designing a control algorithm capable of successfully completing a building-clearing mission, where the environment and object of interest are both unknown. Through exploration of the location, they hope to build virtual maps of traversed areas. The algorithm can be broken down into three parts which are colloborative localization, dynamic task allocation, and collaborative environmental mapping. The first labels the rovers $landmark$ or $moving$, then uses infared sensors to measure and calculate therotical local poses, which are given different weights based on their classification and distance. Factoring individual odometry values into the equation, each pose is published into a goal frame with a particular confidence value. Dynamic task allocation handles the transition from $landmark$ or $moving$ laveled rovers. Since there is no external frame of reference outside of the system, the rovers take turns acting as $landmarks$ while the others move collectively in the goal direction. $Landmarks$ are seen as rovers with high confidence in poses, while $moving$ rovers build confidence as they continual recalculate their position as they pass by $landmarks$. Task allocation occurs when enough confidence has been passed from a $landmark$ to a $moving$ rover at which point the $moving$ becomes a $landmark$, and the $landmark$ to $moving$ if allowed by the system. Finally, the virtual map is constructed by forming triangles within the $2d~plane$ consisting of two $moving$ rovers as the base and the $landmark$ as the point. If the rovers can see each other the are is marked seen by each rover involved, then pushed to a global map. Results were succesful in that rover locationization and mapping was achieve in an unknown environment, with or without obstacles, by augmenting this stepping stone approach.

Although not specific in the implementation, this stepping stone localization approach is interesting. But I believe establishing an anchor node before initial movement would guarantee higher pose confidence. Within this paper, their use of mapping would be the greatest contribution to my research. They have developed a simple, computationally, cost-effective way of building virtual maps on the fly which can be improved on to document resource and obstacles alike.