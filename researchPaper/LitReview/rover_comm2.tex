\textit{Analyzing Myopic Approaches for Multi-Agent Communication}~\cite{becker2009analyzing} \\

In this paper, the authors establish an inter-rover communication threshold, which decides when to communicate information, when constrained by partial rover maps of their overall world. Communication actions are simply $communicate$ or $don't~communicate$, where if one agent chooses to communicate, all agents broadcast their local state to eachother at some cost. This synchronizes the world view generate by the rovers, providing complete information about the current world state. They develop a communication protocol, $Model~Lookahead$, that only requires the rovers to hold the latest message of their projected world, instead of saving the entire message history, which ultimately saves times computationaly. Introducing and controlling messages in this way proved favorable as overcommunication of redundant information was decreased. In two different experimental setups, this approach produced smooth and monotonous degradation of values as communication costs increased. 

This paper wraps up the general notion of computational cost that exists within multi-agent system. In order to implement a succinct, yet complex rover messaging system, like one presented in \cite{ferber1998meta}, inter-rover communications need to be held to a minimum. This will allow the system to be scaleable to excedeling larger size without computational slow down.